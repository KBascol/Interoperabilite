\documentclass[10pt,a4paper]{report}

\usepackage[utf8]{inputenc}
\usepackage{amsmath}
\usepackage{amsfonts}
\usepackage{amssymb}
\usepackage{graphicx}
\usepackage{color}
\usepackage{enumitem}

\usepackage[top=1cm, bottom=2cm, left=2cm, right=2cm]{geometry}
\usepackage{hyperref}
\usepackage{fancyhdr}
\pagestyle{fancy}

\fancyhead{}
\fancyfoot{} 
\lhead{ \hspace{0.1cm} M1 WI 2014-2015  \hspace{0.4cm} \vline}
\chead{Interoperabilité}
\rhead{K.B - B.B - K.L - N.R}
\rfoot{\thepage}

\author{Kevin BASCOL, Bachir BOUACHERIA, Kevin LAOUSSING, Nicolas REYNAUD}
\title{ Interopérabilité: Find Your Way}

\makeatletter
\renewcommand{\thesection}{\@arabic\c@section}
\makeatother

\begin{document}

\makeatletter
	\begin{titlepage}
	
	\centering
		{
		\vspace*{5cm}
		\hrule height 2pt
		\vspace{0.7cm}
		\Huge \textbf{\@title}}\\
		\vspace{0.7cm}
		\hrule height 2pt
		
		\vfill
		\vspace{1cm}
		\@author\\
		\end{titlepage}
\makeatother
\setcounter{secnumdepth}{4}
\setcounter{tocdepth}{3}
\renewcommand{\contentsname}{Sommaire}
\begingroup\makeatletter
\def\@makeschapterhead#1{%
  {\parindent \z@ \raggedright
    \normalfont
    \interlinepenalty\@M
    \Huge \bfseries  #1\par\nobreak
    \vskip 20pt% <---- à réduire pour avoir plus de place
  }}\makeatother
\tableofcontents
\endgroup
\thispagestyle{empty}
\setcounter{page}{0}
\newpage

\newgeometry{top=2cm, bottom=2cm, left=2cm, right=2cm}

%%%%%%%%%%%%%%%%%%%%%%%%%%%%%%%%%%%%%%%%%%%%%%%%%%%%%%%
%%%					INTRODUCTION					%%%
%%%%%%%%%%%%%%%%%%%%%%%%%%%%%%%%%%%%%%%%%%%%%%%%%%%%%%%
\section{Introduction}

\subsection{Description du projet}
\begin{flushleft}
\textbf{Find your way} est un logiciel de type "smart-cities" permettant de localiser toutes les activités à proximité de la position de l'utilisateur, dans un rayon donné pour celui-ci.\\
Parmi les activités localisées par le logiciel, on y trouve les cinémas, les bibliothèques, les bars, et les restaurants. Pour chacune de ses activités, le logiciel est capable de réunir des informations utiles telles que les prix des menus d'un restaurant désigné par l'utilisateur, les horaires des séances d'un film pour un cinéma donné, etc.
\textbf{Find your way} utilise plusieurs sources de données et de formats différents, qui justifie sa composante d'interopérabilité. On y trouve des données de géolocalisation, des données extraites des pages web de sites internet, en plus d'utiliser des données stockées dans une base de données que les utilisateurs peuvent compléter.
Ce rapport présentera dans un premier temps la fonctionnalité de géolocalisation de l'utilisateur et des activités à proximité de celui-ci. Puis dans un deuxième temps, la recherche d'information pour des diverses activités localisés. Dans un troisième temps, une description du système d'introduction d'information par l'utilisateur dans la base de données du logiciel.

\end{flushleft}

\subsection{Outils et supports}

\begin{flushleft}
\textbf{Find your way} est une application web réalisée avec divers outils et supports décrits dans cette liste :

\begin{itemize}

\item API Google Map V3 à retrouver a se lien :
\url{https://developers.google.com/maps/documentation/javascript/}

\item Bootstrap.

\item Langage PHP, javascript, HTML, CSS. 

\item Regex101 qui permet de tester en ligne des expressions régulières.
\begin{verbatim}
	Voir https://regex101.com/ 
\end{verbatim}

\end{itemize}
\end{flushleft}

%%%%%%%%%%%%%%%%%%%%%%%%%%%%%%%%%%%%%%%%%%%%%%%%%%%%%%%
%%%					GEOLOCALISATION					%%%
%%%%%%%%%%%%%%%%%%%%%%%%%%%%%%%%%%%%%%%%%%%%%%%%%%%%%%%

\section{La géolocalisation}

\subsection{Description de la fonctionnalité}
\begin{flushleft}
La géolocalisation nous permet de savoir avec plus ou moins d'exactitude la position d'une personne. Ceci ayant pour but de pouvoir identifier à l'aide des API Google map les lieux à proximité d'une personne. \\

Il s'agit de la base de notre concept, pouvoir localiser une personne.
\end{flushleft}

\subsection{Description technique}
\begin{flushleft}
La géolocalisation se fait à l'aide de l'HTML5 et ses nouvelles fonctionnalités. Nous passons ensuite les informations à gooogle map pour avoir la liste des lieux à proximité de la personne dans un rayon donné. \\
Nous obtenons alors une liste de lieu dans le format Json que nous devons interpréter et en retirer les bonnes informations.\\
Ses lieux sont ensuite affichés sur la carte à l'aide marqueur. Grâce à du Javascript. \\

Pour le calcul de distance, nous avons dû implémenter ceci : 
\url{http://fr.wikipedia.org/wiki/Distance_entre_deux_points_sur_le_plan_cart%C3%A9sien}

Le HTML5 et sa fonction de géolocalisation sont a l'origine de ce pop-up : 
\includegraphics[scale=0.5]{./images/demande.png} \\

A noter : Le pop-up peut être différent en fonction du navigateur.
\end{flushleft}


%%%%%%%%%%%%%%%%%%%%%%%%%%%%%%%%%%%%%%%%%%%%%%%%%%%%%%%
%%%					RECHERCHE INTERNET				%%%
%%%%%%%%%%%%%%%%%%%%%%%%%%%%%%%%%%%%%%%%%%%%%%%%%%%%%%%

\section{Recherche d'information sur internet}

\subsection{Description de la fonctionnalité}
\begin{flushleft}
Lorsque les activités à proximité de l'utilisateur sont localisées, celui-ci a la possibilité de choisir les activités dont il aimerait obtenir plus d'informations.
Par exemple, si l'utilisateur décide d'obtenir des informations sur les horaires des séances des films d'un cinéma, il lui suffit de cliquer sur le cinéma qui l’intéresse et le logiciel se chargera de lui apporter les informations demandées.
\textbf{Find your way} va alors visiter le site web correspondant à l'activité ( dans l'exemple d'une activité de type cinéma, il va consulter le site \url{http://www.allocine.fr/} ) et extraire les informations des pages web du site.\linebreak

Les sites web consultés par \textbf{Find your way} en fonction de l'activité :

\begin{itemize}
\item Cinéma : \url{http://www.allocine.fr/}.

\item Livre : \url{http://livre.fnac.com/}.

\item Restaurant et bar : leurs propres sites web ( si ceux-ci sont référencés sur internet).

\end{itemize}
\end{flushleft}

\subsection{Description technique}
\begin{flushleft}
Pour extraire l'information des pages web, \textbf{Find your way} un programme PHP qui va récupérer la page web en tant que chaîne de caractère. Puis il utilise des expressions régulières pour filtrer et trouver les bonnes balises HTML contenant les informations à extraire.\newline

Exemple :
 Sur ce \href{http://www.allocine.fr/seance/salle_gen_csalle=P0231.html}{lien}, \textbf{Find your way} cherche à récupérer tous les titres des films à l'affiche. \\
 \includegraphics[scale=0.45]{./images/allocine_img1.png}
 
 Pour se faire il analyse le code source de la page html et reconnaît la chaîne de caractères (cadre bleu)  et récupère le titre (cadre rouge)\\
  \includegraphics[scale=0.45]{./images/allocine_img2.png}
 avec cette expression régulière : \\
 \includegraphics[scale=0.45]{./images/allocine_img3.png}
\end{flushleft}

%%%%%%%%%%%%%%%%%%%%%%%%%%%%%%%%%%%%%%%%%%%%%%%%%%%%%%%
%%%					BASE DE DONNEE					%%%
%%%%%%%%%%%%%%%%%%%%%%%%%%%%%%%%%%%%%%%%%%%%%%%%%%%%%%%

\section{Base de donnée}

\subsection{Description de la fonctionnalité}
\begin{flushleft}
Pour ajouter des lieux et avoir le suivi des votes, nous devons stocker tout ceci dans la base de données afin de pouvoir retourner les informations en temps voulu. \\

Les votes sont stockés par Ip et par article. \\
\end{flushleft}

\subsection{Description technique}
\begin{flushleft}
Pour effectuer les votes, nous avons dû, à l'aide du php, récupérer l'ip de la personne qui vote. Ainsi nous pouvons limiter le nombre de vote par ip et par lieu. \\

Pour ajouter un lieu, nous récupérons la positon de la personne, puis encore une fois à l'aide des API google map qui inverse la géolocalisation ( des coordonnées donnent une adresse ). Nous pouvons alors ajouter le lieu en base de données avec le nom fourni par l'utilisateur. \\

Nous avons du alors faire une corrélation entre les différentes tables suivante :
\includegraphics[scale=0.5]{./images/BDD_Vote_Lieu.png}

Ici nous avons les 2 champs lieu qui sont les mêmes mais également le champs id.VoteIp égale au champs id\_lieu.Lieu. \\


Ainsi il existe une corrélation entre toutes nos tables. \\

Lors de l'affichage nous devons encore une fois implémenter 
\url{http://fr.wikipedia.org/wiki/Distance_entre_deux_points_sur_le_plan_cart%C3%A9sien}

puisque nous devons savoir à combien de mètres nous sommes du nouveau lieu.
Si le lieu est dans notre périmètre alors nous l'affichons.\\
\end{flushleft}


%%%%%%%%%%%%%%%%%%%%%%%%%%%%%%%%%%%%%%%%%%%%%%%%%%%%%%%
%%%						CONCLUSION					%%%
%%%%%%%%%%%%%%%%%%%%%%%%%%%%%%%%%%%%%%%%%%%%%%%%%%%%%%%

\section{Conclusion}
\begin{flushleft}
Pour la création de cette application web nous avons eu l'occasion de récupérer des informations provenant de diverses sources. Des sources provenant d'Internet (google, la fnac, allociné) mais aussi provenant de notre propre base de données. Toutes ces extractions de données forment les composantes d'interopérabilité que nous avons mis en œuvre dans le cadre de ce projet.
\end{flushleft}
\end{document}
